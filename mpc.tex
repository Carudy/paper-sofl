\documentclass[journal]{IEEEtran}

\usepackage{amsmath}
\usepackage{url}
\usepackage{graphicx}
\usepackage{algorithm}
\usepackage{algorithmicx}
\usepackage{algpseudocode}

\title {SOFL: Self-organizing Federated Learning Based on Multi-party Computation and Consensus Algorithm}
\author{Zhaoyang Han}


\begin{document}
\maketitle

\begin{abstract}
With the rapid development and widespread application of artificial intelligence (AI) technologies, privacy and security are attracting more attention in machine learning areas. Federated Learning (FL) is a framework based on which numerous parties can train a machine learning model cooperatively without leaking information about their data. Parties in FL send parameters of local models instead of their data to the server in order to obtain the global model securely. However, researches have proven that attackers can reveal FL parties' data through the leakage of local model parameters. Therefore, secure aggregation is requisite to ensure privacy-preserving in FL frameworks. Moreover, it usually costs a lot for FL parties to securely communicate with each other. Therefore, efficiency is another problem for FL to address. In this paper, we propose a privacy-preserving FL framework based on secure multi-party computation (MPC) and simplified consensus algorithm, namely Self-organizing Federated Learning (SOFL). Our method adopts secret sharing to hide local model parameters, and runs a consensus algorithm to ensure the communication is reliable, robust and efficient.

\end{abstract}

\begin{IEEEkeywords}
    Federated Learning, Secure Aggregation, Machine Learning, Multi-party Computation, Consensus algorithm
\end{IEEEkeywords}


\section{Introduction} 
\label{sec:intro}
Smartphones and other devices in daily life are equipped with more and more powerful computing and storage abilities, which enables individual devices to accumulate more and more valuable information. It facilitated a lot of rising technologies such as edge computing. Meanwhile, the accumulated data in users' devices can be used to train models for various practical purposes due to the flourishing of machine learning. In traditional machine learning frameworks, data needs to be gathered in a central server in order to execute the learning process. However, most data collected by mobile devices is sensitive. Users usually refuse to send their private data to others, such as a learning center. 

Federated Learning\cite{mcmahan2016communicationefficient} is designed to address this problem. In each round, FL parties receive a global model from the server, and they train their model based on their own data respectively. Afterwards, the parties send the parameters of their models to the server while the server runs a particular algorithm to compute the global model based on these parameters. In such frameworks, users don't need to send their data to the learning server, which can protect privacy to some extent.

However, many researchers have shown that attackers are able to infer users' data through the leaked information about the model's parameters\cite{Beyond, Leakage}. Therefore, sending parameters to the server directly is no longer secure. Secure aggregation protocols allow a group of parties who have private information to compute a function that takes these private values as parameters. Researchers paid their attention to secure aggregation for a better solution\cite{shi2011privacy,RobustAgg,Bonawitz19,Nike,PrivFL}. There are 3 primary methods to achieve privacy-preserving in secure aggregation: Differential Privacy (DP), homomorphic encryption (HE), and secure multi-party computation. DP enabled FL focuses on provides privacy-preserving while keeping the accuracy of the machine learning model. Since these studies did not conduct experiments with complicated machine learning models while some researches have pointed out that DP based FL will impact the accuracy of the learned model\cite{Two-Phase}, it is still a challenge to employ DP in FL. Homomorphic encryption algorithms are intuitionistic and simple to protect privacy, however, they suffer from low efficiency which is hardly acceptable in FL\cite{HESurvey}. Blockchain-based methods\cite{DeepChain,Lu2020,On-Device} are also very promising, and the generally used consensus algorithms in them can be inspiring. Yet blockchain-based methods are still implement-unfriendly. Therefore, adopting MPC to protect users' privacy is more practical. Many types of research are protecting the parameters based on MPC\cite{Practical,Two-Phase,Weighted,Hybrid}. 

Secure multi-party computation can be implemented by garbled circuits or secret sharing methods\cite{Shamir}. Garbled circuits have many limits and low efficiency. Therefore, we choose to use secret sharing methods. Normally, secret sharing needs parties to exchange information among themselves. However, in federated learning frameworks, the parties are usually strange to each other, which means one party does not have the addresses of others. A party cannot communicate with other parties directly and they can only exchange information securely with the help of the server. In this case, Bonawitz etc.\cite{Practical} proposed a method about constructing secure channels among FL parties. Constructing secure channels between every pair of parties cost plenty of time, which leads to a new problem. In addition, the robustness of FL frameworks is also significant because it usually costs a lot to recover from situations that several nodes are crashed. In summary, employing traditional MPC methods in a system with a large number of users is faced with a problem with efficiency and instability.

\textbf{Our contribution: }In this paper, we propose Self-organizing Federated Learning (SOFL), a novel FL framework that utilizes MPC to protects users' privacy and takes advantage of consensus algorithms to achieve high efficiency and robustness. Our model first elects some leaders, who will construct secure communication with other parties. Afterwards, a party only needs to exchange information with the leaders. The leaders will send the received information to the server, who helps to forward the information to the corresponding destinations. Appointing leaders reduces the need for communications greatly and running a consensus algorithm can handle unexpected situations where a leader node or a common client is crashed. We adopted a simple additive secret sharing protocol to realize MPC. Our model also has strong robustness based on the consensus algorithm due to the consensus algorithm. 

\textbf{Roadmap:} In Section~\ref{sec:back} we introduce the background of knowledge and some definitions. Next, we introduce related work and some platforms of federated learning in Section~\ref{sec:related}. Section~\ref{sec:sofl} detailedly illustrates our proposed framework while describes the attack model. Evaluations for efficiency and security are stated in Section~\ref{sec:eval}, followed with experiments and results in Section~\ref{sec:exp}. Finally, we give the conclusion and future expectations in Section~\ref{sec:conc}.

\section{Background}
\label{sec:back}
In this section, we introduce the concepts of key exchange, Federated Learning, multi-party computation, and consensus algorithms, which are used in our proposed framework.

\subsection{Key Exchange}
Key exchange protocols allow several parties to share secret keys under unsafe conditions. Diffie-Hellman~\cite{DH} (DH) is a prestigious key exchange protocol. We introduce how DH protocol helps two parties to achieve agreement: suppose $a$ and $b$ want to obtain a secret key by means of DH. First they select a group $G$ of order q, and randomly choose a number $x_a$ and $x_b$ respectively. Then $a$ calculates $g_a = g^{x_a}$, where $g$ is a generator of $G$, and $b$ calculates $g_b = g^{x_b}$. $a$ sends $g_a$ to $b$, and $b$ sends $g_b$ to $a$. Finally they can calculate the shared secret $s_{ab}$ respectively:
$$ s_{ab} = g_b^{x_a}  = g_a^{x_b} = g^{x_ax_b}$$

In this scheme, only $g_a$ and $g_b$ are sent under an unsafe condition. Attackers cannot infer $s_{ab}$ from $g_a$ and $g_b$, therefore $a$ and $b$ can communicate privately by means of $s_{ab}$. DH is lightweight and efficient, and it can be expanded to multi-party versions easily in order to enable more parties to share pairwise keys.

\subsection{Federated Learning}
FL enables a number of users to jointly train a model. In each round of FL, each user will train the model based on its data and get the model's parameter. Denote the $i$-th user's local parameter in the $t$-th round by $W_i^t$. When all users have trained their model in the $t$-th round, an aggregation algorithm $Agg$ will be called to compute the global model's parameter $Agg(W_1^t, W_2^t, ..., W_n^t)$, where $n$ is the total number of users. For example, weighted average was adopted as $Agg$ in FedAvg~\cite{mcmahan2016communicationefficient}. FL not only helps to protect users' privacy but also deals with the ``data in form of isolated islands'' problem for companies or institutions. Yang \emph{et al}. also categorized FL into horizontal, vertical, and hybrid styles based on the fact that whether the data shares the same feature space or entities~\cite{yang2019federated}. 

Federated Learning can be generalized to two work environments:

\begin{enumerate}

    \item \textbf{Among-institutions:} In this situation, FL is usually used to help companies or other institutions solving the ``Isolated Data Island'' problem, where data of different institutions are with different distributions and even different categories of features. This is a severe problem in traditional machine learning, while FL can solve these kinds of problems by the well-designed aggregation algorithm. 

    Generally, companies and institutions are responsible social individuals. Therefore, we can suppose Public Key Infrastructure (PKI) is already enabled, and one party can communicate with any other one privately, i.e., P2P communication is already enabled in this environment.

    \item \textbf{Server-based:} In this case a company or institution aims to train a model based on their users' data while protecting their privacy. The communications of FL parties always need to go through the server. Under this situation, all information may be eavesdropped on by the server if it is untrusted. 

    However, the users of an institution usually do not know each other. It means that two parties can hardly exchange information privately without the help of a server. Therefore, the confidence problem is severe in such environments.

\end{enumerate}

Figure~\ref{fl_sit} illustrates the structures of two environments. Executing MPC protocols and consensus algorithms are at a low price in the first environment. In the second environment, if local parameters need to be protected, we need to run some key-exchange protocols first to construct secure channels among users. One of our contributions is that we solved the problem in the second environment by building connections selectively and efficiently.

\begin{figure}[!ht]
    \centering
    \subfloat[Among-institutions]{\includegraphics[width=2.5in]{img/fl_sit_ins.eps}%
    \label{fl_sit_institution}}
    \hfil
    \subfloat[Server-based]{\includegraphics[width=2.5in]{img/fl_sit_server.eps}%
    \label{fl_sit_server}}
    \caption{The structures of two work environments in FL. The first is the ``Among-institutions'' model where institutions can communicate with others directly, and the second is ``Server-based'' model where parties exchange information in virtue of the server.}
    \label{fl_sit}
\end{figure}

\subsection{Multi-party Computation}
Secure MPC is a branch of cryptography which enables several parties to compute a particular function without leaking their own data (inputs). Suppose there are $n$ parties employing MPC to compute a function $F$, and the $i^{th}$ party has its parameter $A_i$. Their goal is to compute $R = F(A_1, A_2, ..., A_n)$. MPC has the feature that participants can only obtain $R$ from the process and the party $P_i$ has no idea about parameter $A_j (j \ne i)$. This feature fits the aggregation algorithm in FL greatly.

Most MPC protocols depend on two cryptography technologies: secret sharing~\cite{Shamir} and oblivious transfer~\cite{OT}. MPC can be implemented with garbled circuits, multi-party circuit-based protocols, or hybrid methods~\cite{mpc-sok}. It also benefits from fully homomorphic encryption (FHE) algorithms. Garbled circuits and FHE suffer from complicated design and poor efficiency. Therefore, secret sharing methods are more favored to solve the privacy-preserving problem in FL. A secret sharing scheme involves a secret $s$, a set of $n$ parties, and a collection $A$ of subsets of parties. Each party has its share of $s$. The secret sharing scheme ensures any subset in $A$ can reconstruct $s$~\cite{Secret-Sharing-survey}. 

SPDZ~\cite{SPDZ} (speedz) is a practical and secure secret-sharing-based MPC protocol introduced by Damgard \emph{et al}. It supports addition and multiplication by means of the triples~\cite{Triple}, which are generated by somewhat homomorphic encryption (SHE). Our method does not require multiplication and hence we do not need to generate the triples. We adopt the resharing method used in SPDZ to achieve secure addition in DemoFL, which is essential in the aggregation phase of FL.


\subsection{Consensus Algorithms}
In a distributed or multi-party system, there is always a problem with consensus, i.e., in such systems parties always need to achieve agreement on a certain value. This could be difficult without any strategy because different parties may be in different statuses and have multifarious matters. Consensus algorithms are adopted to address such problems. It is widely used in blockchain and various famous areas.

Paxos was the first consensus algorithm introduced by Lamport~\cite{Paxos}. It helps the nodes of a cluster to select several leaders democratically and reach a consensus with the help of these leaders. Paxos is used in a lot of famous projects such as Ceph~\cite{Ceph}. Raft is a modification of Paxos which is more implement-friendly~\cite{Raft}. It contains two phases: leader election and log replication. Parties can achieve agreements based on leaders. Considering that FL models are semi-decentralized systems, our framework can utilize Raft algorithm to select several leaders, based on which MPC protocols can be executed efficiently and robustly.

\section{Related Work}
\label{sec:related}
Homomorphic encryption based solutions are intuitively effective to solve aggregation problems, and there are many researchers tried to reduce the overhead caused by HE. Stephen et al.~\cite{abs-1711-10677} designed an additively homomorphic encryption method for federated learning in 2017. However, the encryption method is complicated that it does not reach high efficiency. The state-of-the-art HE method was proposed by Zhang and Li~\cite{BatchCrypt}. It reduced the encryption overhead greatly at the cost of trivial loss of accuracy. However, it still costs much time on computation compared to MPC based and DP based methods.

DP based solutions have better performance than HE based solutions, while Wei et al.\cite{DPAnalysis} has indicated that there is a tradeoff between the performance and security levels, which means it needs numerous adjustments to adopt DP methods efficiently. Robin et al~\cite{geyer2017differentially}. tested DP in FL in 2017 and the result showed that DP's influence on accuracy is untrivial. Bayesian differential privacy~\cite{Bayesian} was proposed in 2019. It considered the probability and distribution of data and is effective for machine learning models whose data are often restricted to a particular type. While it may be inefficient in vertical federated learning situations. Whereas all these methods did not prove DP does not impact the accuracy in other more complicated and large-scale models, which means it is different to find a universal DP method for all machine learning models.

MPC based methods have the least computation cost but require more communications. A typical MPC federated learning model is implemented by Google~\cite{Practical}. It requires pairwise key exchange among all clients, which results in enormous overhead on communication. It also provides robustness by means of mask mechanism with some random numbers. However, these mechanisms bring about more overhead. Other researches present hybrid methods~\cite{Hybrid,HybridAlpha}. These hybrid methods combine MPC with either HE or DP and make tradeoff on computation and communication.

Federated learning is more and more practical nowadays and there are already many FL platforms: FATE~\cite{fate} is an open-source federated learning project proposed by Webank’s AI Department. It adopts both MPC and HE to implement secure aggregation, while it is still absorbing state-of-the-art methods for privacy-preserving. Pysyft~\cite{pysyft} is another open-source federated learning framework presented by OpenMined. It is based on Pytorch and offers HE, DP, and MPC as alternative methods to realize privacy-preserving. However, the cost of time of Pysyft is dozens of times than pure Pytorch, which indicates that privacy-preserving methods of current federated learning platforms need to be improved.



\section{Self-organizing Federated Learning}
\label{sec:sofl}
\input{sec/sofl}


\section{Efficiency and Security Evaluation}
\label{sec:eval}
\subsection{Communications}
As illustrated in Section III, our framework only adopts lightweight algorithms to enhance security. Therefore, the execution overhead can be ignored compared to the communication overhead. Notice that any communication in our framework goes through the server. E.g., if $P_i$ wants to send encrypted message $c$ to $L_j$, $c$ will be sent to server $S$ first. Afterwards $S$ will forward $c$ to $L_j$. In general, our P2P communications are emulated with client/server communications. This method accelerates communication greatly because it costs much for two strangers to exchange messages directly. E.g., if two users want to communicate directly, both of them need to store the addresses, confirm the ``accept'' signal after each message-exchange, \emph{et al}. However, with a powerful server helping to forward, these things are no longer concerns for users.

In each round, there are $n$ common users and $N_l$ leaders. To analyze the communication overhead clearly and without loss of generality, we can suppose there is no common user that is also a leader at the same time. Common users, leaders together with the server $S$ are all users requiring communication. Generally, $N_l$ is a very small number such as 3, therefore we can treat it as a constant number. Our framework can be categorized into \textbf{set-up} and \textbf{secure-learning} two phases. We analyze these two phases respectively:

\begin{itemize}
    \item \textbf{Set-up:} First, all $n$ users will send ``self-recommendation'' messages to decide who the leaders are, then $n - N_l$ common users need to construct secure channels with all $N_l$ leaders. Afterwards, the server will send $L$ to all $n$ users. Suppose it needs $D$ communications in each DH protocol, then the amount of communications of the set-up phase is $n + n + D * (n - N_l) * N_l$. The time complexity is $O(n)$ based on the fact that $D$ and $N_l$ are small constant numbers.
    
    In the reorganization process, $n - N_l$ common users will send ``self-recommendation'' messages to decide a new leader. The new leader then conducts DH protocols with all $n - N_l - 1$ users, which is $D * (n - N_l - 1)$ communications. Therefore, a reorganization process also costs $O(n)$ communications.
    
    \item \textbf{Secure-learning:} In each epoch, $S$ sends the current $W_\textrm{global}$ to $n$ users, which cost $n$ communications. Then each common user sends $W_{ij}$ to $N_l$ leaders respectively, which cost $n * N_l$ communications. Afterwards, each leader sends $B_j$ to the server and waits for the intersecion, which cost $2 * N_l$ communications. Finally leaders send $Aj$ to $S$, which cost $N_l$ communications. Thus the total cost of one epoch is $n + n * N_l + 3 * N_l$, which is $O(n)$.
\end{itemize}

In the time complexity aspect, our framework does not result in higher overhead expect for an $O(n)$ preprocessing compared to the original FedAvg algorithm. In the vertical aspect, our framework does not require more message-exchanges for any party-leader pair. 

\subsection{Security Evaluation}
The evaluation is based on the fact that our adversaries are all honest-but-curious. Since our framework is based on MPC researches, the proof of security against message-leakage can be referred~\cite{Shamir,Du2001SecureMC,Three-Party}. And the security of using MPC in FL has been proved by Zhu \emph{et al.}~\cite{Weighted} The riskiest threat is the collusion attack. Colluding with a common party has no contribution to an attack because it lets out nothing but the information about this common party, which belongs to the attacker side. Therefore, we only discuss collusions among the server and leaders. 

Apparently, the attacker must collude with all leaders in order to reconstruct one party's parameter. Since the leaders are assigned randomly based on the consensus algorithm, it is hardly possible for all attackers to be insincere at the same time. However, it is necessary to discuss the situation where the server cheats to select leaders as its wish by other means such as slowing down the sincere users' network. In such situations, the leaders are always selected by the unreliable server. To address this problem, we introduced ``leader-tenure'' in Section III, which forces the system to change leaders regularly. Since it needs all leaders' betrayal to attack successfully, the system only needs to change one leader regularly. Changing one leader is equivalent to a reorganization process, whose time cost is $O(n)$. Therefore, our framework has high security against collusive honest-but-curious attackers.



\section{Experimental Results}
\label{sec:exp}
\subsection{Implementation}
Our framework is implemented with Pytorch. We used AES-CFB-128 as the authenticated encryption algorithm as Bonawitz \emph{et al}. did~\cite{Practical}. We adopted MNIST and CIFAR-10 as datasets, which are also used in the original FedAvg~\cite{mcmahan2016communicationefficient}. We trained a simple convolutional neural network for the classification task. Our experiments are carried on a PC with an Intel i7-8700 CPU (3.2GHz), 16 GB of RAM, and a GTX 1080 GPU. The model was executed in a single thread to facilitate comparing and analyzing. The optimizer was stochastic gradient descent (SGD) and the learning rate is $0.01$. The $fraction$ is set as $0.1$ which means that $10\%$ of clients will be chosen to carry on the training process in each epoch. 
% Our demo is open-sourced on \url{https://github.com/Carudy/DemoFL}.

\subsection{Accuracy}
Although our work does not modify the learning module compared to other federated learning frameworks, we still conducted a series of experiments to observe the accuracy. We compared FedAvg with our framework on both independently identically distribution (IID) data and non-IID data to verify the effectivity. Since this experiment aims to prove the validity instead of high accuracy, the models were not trained to obtain high accuracy. We conducted experiments with MNIST in non-IID and CIFAR-10 in IID, and Figure~\ref{acc} shows the result. With IID data, our framework obtains a similar performance compared with FedAvg. With non-IID data, the two systems did not match as well as they were with IID data when the $epoch$ was small. However, they finally converged to the same stable scope with the same speed, which confirmed the differences were caused by biases. Therefore, our framework can achieve the same learning goal as FedAvg while providing privacy guarantees and system robustness.

\begin{figure}[!ht]
    \centering
    \includegraphics[width=\columnwidth]{img/acc.eps}
    \caption{The accuracy of FedAvg~\cite{mcmahan2016communicationefficient} and DemoFL with IID data from CIFAR-10, and non-IID data from MNIST. The horizontal axis $epoch$ means the total rounds that the federated learning system has been trained for. }
    \label{acc}
\end{figure}

\begin{figure}[!ht]
    \centering
    \includegraphics[width=\columnwidth]{img/leader-time.eps}
    \caption{The average time spent on computation in set-up process DemoFL VS different proportions of leaders.}
    \label{leader-time}
\end{figure}

\subsection{Set-up Overhead}
In the set-up phase, our system selects several clients as leaders, and the number of leaders impacts the efficiency and a particular leader's load because a common user needs to build secure communication channels with all leaders. To discover the relation between the number of leaders and the computation overhead, we set the number of clients on different levels and conducted experiments with different numbers of leaders. Figure~\ref{leader-time} shows the linear relationship between each user's computation time spent on Diffie Hellman key-exchange protocol and the number of leaders. The overhead of time in the set-up process is also very low because DemoFL has no complex calculations as Bonawitz \emph{et al}.~\cite{Practical}'s secure aggregation scheme does. In Bonawitz's scheme, participants need to calculate secret keys for all other users and generate t-out-of-n secret shares for the pseudorandom generator (PRG) seed and their secret keys. The comparison is illustrated in Figure~\ref{avg-user-cpu}. Since Bonawitz's scheme has a time complexity of $O(n^2)$, the computation overhead of it is much higher than DemoFL. Method of Kanagavelu \emph{et al}.~\cite{Two-Phase} performs the same as Bonawitz's scheme because they also have an $O(n^2)$ set-up. Therefore, employing fewer leaders can help to improve efficiency, and in contrast, employing more leaders can are beneficial to security because a successful attack requires all leaders compromised. 

Meanwhile, with more leaders, one common user needs to store more keys for leaders. Apparently, the relationship between the number of leaders and one user's storage overhead is also linear. However, in Bonawitz's scheme, a client needs to storge two t-out-of-n secret shares for all others. The storage complexity of DemoFL and Bonawitz's scheme is $O(N_l)$ and $O(n)$ respectively, where $N_l \ll n$. Therefore, DemoFL has a very trivial set-up overhead on both time and storage, which is friendly to individual participants such as smartphones.

\begin{figure}[!ht]
    \centering
    \includegraphics[width=\columnwidth]{img/avg-user-cpu.eps}
    \caption{The average time spent on computation in set-up process of Bonawitz's secure aggregation~\cite{Practical} and DemoFL VS the number of clients. The proportion of leaders is set to $3\%$.}
    \label{avg-user-cpu}
\end{figure}

\subsection{Efficiency}
Our framework primarily addresses the server-based situation, where a participant's computing capability is far weaker than the server. And for the server, the overhead of computations such as MPC is insignificant compared to the communication overhead because of its powerful computing capability. Therefore, we carried out experiments to evaluate the computation overheads for participants only, and communication overheads for both participants and servers in DemoFL and Bonawitz's secure aggregation scheme. Experiments in this subsection are with the assumption that there are no dropouts or crashes.

To evaluate the computation overheads, we fixed the number of clients to 100 and the proportion of leaders to $3$ and tested the average running time of a single participant. In DemoFL, participants are divided into two categories: common users and leaders, and a participant can hold both identities at the same time. While in Bonawitz's secure aggregation scheme, all users are treated equally. The result is displayed in Figure~\ref{learning-overhead}. It suggests that participants of DemoFL have obvious advantages on computation overhead. The reason is that participants of Bonawitz's secure aggregation scheme need to carry out more encryptions/decryptions for unmasking.

Measuring the communication overheads with wall clock time is difficult because it has biases large enough to mislead the judgment. Therefore, we evaluate the communication overhead with the number of communications. We conducted experiments to count up the communications and tried to compare the overheads of DemoFL and Bonawitz's secure aggregation scheme. Since there are no dropouts or crashes, the number of communications in each epoch is fixed. Therefore, we computed the ratio of overheads of Bonawitz's secure aggregation scheme and DemoFL with the number of clients set as $100$ and the number of leaders as $3$, and the result is $1.184$. And if the number of leaders gets larger, this ratio will decrease towards $1$. Since the number of leaders will never be set too large, the communication overhead of DemoFL is advantageous.

\begin{figure}[!ht]
    \centering
    \includegraphics[width=\columnwidth]{img/learning-overhead.eps}
    \caption{The average running time per participant in the learning process VS epoch. Without dropouts and reorganization processes.}
    \label{learning-overhead}
\end{figure}

\subsection{Robustness}
\subsubsection{Crash}
The robustness of DemoFL is mainly based on the reorganizing process, which happens when a leader is crashed. We discuss crashes for leaders because a crash of a common user is equivalent to a dropout. Crashes can hardly be completely avoided, therefore, we introduced $crash\_rate$, which is the possibility for each leader that it would crash during one epoch, to help to measure the robustness. Generally, we consider $crash\_rate$ is quite small because compared to dropouts, crashes rarely happen in nowadays smart devices/servers. Therefore we can consider $crash\_rate$ would not be larger than $10\%$. Since a crash causes a reorganizing process, which helps the system by communications, we still adopt the number of communications to evaluate the extra overhead. We conducted several experiments on how $crash\_rate$ impacts efficiency. First, we set the number of clients to 100 and the $crash\_rate$ to $10\%$, which is the worst case. Then we counted up the number of communications at a situation without crashes. Finally, we counted up communications in other situations with different numbers of clients and different numbers of leaders. We consider the ratio of communications in these situations with the original one as their overheads, and the result is shown in Figure~\ref{crash-leader}. 

\begin{figure}[!ht]
    \centering
    \includegraphics[width=\columnwidth]{img/crash-leader.eps}
    \caption{The ratio of communications VS number of clients and leaders.}
    \label{crash-leader}
\end{figure}

The number of clients hardly affects the overheads. In contrast, the number of leaders is very influential. Therefore, we suggest setting the number of leaders as smaller as possible. Generally, $3$ or $5$ could be sufficient choices.

\begin{figure}[!ht]
    \centering
    \includegraphics[width=\columnwidth]{img/dropout-acc.eps}
    \caption{The accuracy VS different $dropout\_rate$s.}
    \label{dropout-acc}
\end{figure}

\subsubsection{Network Delay and Dropout}
Sometimes several packets cannot be received in time due to network delays or dropouts. Bonawitz \emph{et al}.'s method~\cite{Practical} has considered this issue and it has a ``double-masking'' scheme to address it, which has a high computation overhead. In DemoFL's secure learning process, a leader will not be waiting for participants'parameters permanently. It has a time limit, over which the leader will abandon waiting and send the current $B_j$ to the server (introduced in Section~\ref{sec:DemoFL}). Under this circumstance, some common parties will be recognized as having lost connection and removed from the participant-group without contributing to the learning process. However, if a party does not lose the connection while its message reached the leader late because of network delay, discarding it may influence the target model's accuracy. Since the leaders will compute the intersection no matter there are dropouts or not, the dropouts have no negative effect on efficiency. We set the $dropout\_rate$ to show the probability that a common party fails to send its parameters to leaders due to packet loss or network delay in one epoch. We then carried out several experiments to observe how package losses impact the learning process. The result is shown in Figure~\ref{dropout-acc}. The result indicates that the influence of dropouts on accuracy is acceptable when $dropout\_rate$ is no more than $10\%$: when the $Epoch$ gets larger than about $160$, the accuracy gap between no-dropout models and with-dropout models becomes insignificant. In addition, the accuracy of the model learned in the situation with $5\%$ $dropout\_rate$ is even better than the model learned in the situation without $dropout\_rate$,  which may be caused by biases, or the inherent defects of federated learning's aggregation scheme, and this might be one of our future researches. In summary, DemoFL has a strong resistance to packet loss considering that most machine learning tasks have a large enough $Epoch_{max}$.


\section{Conclusion and Future Works}
\label{sec:conc}
We proposed a novel, efficient and privacy-preserving federated learning framework named Self-organizing Federated Learning (SOFL). SOFL adopted MPC-based secret sharing methods to solve the secure aggregation problem and achieve high security. It also employ consensus algorithms-based leader election algorithms to provides high efficiency and robustness. SOFL is lightweight and easy-deploying for many existing federated frameworks.

However, it still remains a problem for federated learning, that malicious attackers are more harmful and difficult to defend against compared to honest-but-curious attackers. Some blockchain methods are inspiring to solve the malicious attacks such as poisoning and adversarial examples, which can be adopted to enhance SOFL to provide higher security. Federated learning is potential and will make a better life for humans.




\bibliography{mpc.bib}
\bibliographystyle{IEEEtran}
\end{document}