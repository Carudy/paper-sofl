In this paper, we proposed a novel, efficient and privacy-preserving federated learning framework named Self-organizing Federated Learning (SOFL). SOFL adopted MPC-based secret sharing methods to achieve privacy-preserving against honest-but-curious adversarials. It also employs consensus algorithms based leader election algorithms to reduce the communication cost and provide higher efficiency and robustness. Our experiments showed that SOFL has a linear time complexity VS either number of clients or training rounds. And it has a high resistance to member changes. SOFL is lightweight and easy-deploying for many existing federated frameworks such as Pysyft and FATE.

However, it still remains a problem for federated learning that malicious attackers are more harmful and difficult to defend against compared to honest-but-curious attackers. A malicious attacker can deviate from designed protocols or cheat on data that would go through it. Enhanced MPC protocols such like SPDZ\cite{SPDZ} may help dealing with malicious attackers, which could be a future work. Therefore, detecting malicious nodes is a challenging work. In addition, other federated learning schemes besides FedAvg may contain algebraic calculus more than only addition and multiplication, which means existing MPC based methods will gain high overhead on those schemes. Some blockchain techniques are inspiring to solve the malicious attacks such as poisoning and adversarial examples, which may be adopted to enhance SOFL to provide higher security by preventing frauds. It would be significative and interesting for a future work to find more approaches about improving the security and efficiency of federated learning.